\documentclass[11pt]{beamer}
\usepackage[UTF8]{ctex}
\usepackage{comment,caption}
\usepackage{tikz,verbatim}
\usepackage{wrapfig}
\usepackage{graphicx,subfig}
\usepackage{lipsum}
\usepackage{mwe}
\usepackage{xeCJK}
\usepackage{fontspec}
\usepackage{booktabs}
\usepackage{array}
\usepackage{colortbl}
\setCJKmainfont[AutoFakeSlant=true, AutoFakeBold=2]{STSong}
\newcommand{\PreserveBackslash}[1]{\let\temp=\\#1\let\\=\temp}
\newcolumntype{C}[1]{>{\PreserveBackslash\centering}p{#1}}
\newcolumntype{R}[1]{>{\PreserveBackslash\raggedleft}p{#1}}
\newcolumntype{L}[1]{>{\PreserveBackslash\raggedright}p{#1}}
\linespread{1.0}
\definecolor{darkgreen}{rgb}{0.0, 0.2, 0.13}
\definecolor{aggressiveRed}{RGB}{204, 0, 0}   % 進攻型顏色
\definecolor{defensiveBlue}{RGB}{0, 51, 102}  % 防守型顏色
\definecolor{tableGray}{RGB}{240, 240, 240}

\mode<presentation>
{
  \usetheme{Berlin}      % or try Darmstadt, Madrid, Warsaw, ...
  \usecolortheme{beaver} % or try albatross, beaver, crane, ...
  \usefonttheme{serif}  % or try serif, structurebold, ...
  \setbeamertemplate{caption}[numbered]
  \setbeamertemplate{navigation symbols}{}
} 


\newtheorem{conjecture}[theorem]{Conjecture}
\newtheorem{proposition}[theorem]{Proposition}

\setbeamertemplate{blocks}[framed]
\setbeamercolor{block title}{fg=blue, bg=gray!40}
\setbeamercolor{block body}{fg=black,bg=gray!10}
\setbeamercolor{block title alerted}{fg=red,bg=gray!40}
\setbeamercolor{block title example}{fg=black,bg=green!20}
\setbeamercolor{block body example}{fg=black,bg=green!5}
%\setbeamerfont{block title}{series=\bfseries}
% --- 多作者 ---
\newcommand{\myauthor}[2]{%
    \parbox{0.22\textwidth}{%
        \centering
        #1 \\ 
        \scriptsize #2 
    }%
}

\usepackage[english]{babel}

\makeatletter

% --- 标题页 ---
\title[中欧时代先锋股票分析] 
{中欧时代先锋股票基金经理业绩评估} 
\author[董安愉、李想、杨晨艺]
{
    \myauthor{董安愉}{Anyu Dong}\hfill 
    \myauthor{李想}{Li Xiang}\hfill
    \myauthor{杨晨艺}{Chenyi Yang}
}
\institute[] 
{\large 北京大学元培学院}
\date{Novenmber 25, 2025}

% --- 自定义页脚 ---
\setbeamertemplate{footline}
{
  \leavevmode%
  \hbox{%
  % 左边:作者 (单位)
  \begin{beamercolorbox}[wd=.333333\paperwidth,ht=3ex,dp=1.5ex,center]{author in head/foot}%
    \usebeamerfont{author in head/foot}\insertshortauthor
  \end{beamercolorbox}%
  % 中间:标题
  \begin{beamercolorbox}[wd=.333333\paperwidth,ht=3ex,dp=1.5ex,center]{title in head/foot}%
    \usebeamerfont{title in head/foot}\insertshorttitle
  \end{beamercolorbox}%
  % 右边:日期 页码
  \begin{beamercolorbox}[wd=.333333\paperwidth,ht=3ex,dp=1.5ex,right]{date in head/foot}%
    \usebeamerfont{date in head/foot}\insertshortdate{}\hspace*{2em}
    \insertframenumber{} / \inserttotalframenumber\hspace*{2ex} 
  \end{beamercolorbox}}%
  \vskip0pt%
}

\makeatother

% --- 自定义页眉 ---
\setbeamertemplate{headline}
{
  \begin{beamercolorbox}[colsep=1.5pt]{upper separation line head}
  \end{beamercolorbox}
  \begin{beamercolorbox}{section in head/foot}
    \vskip2pt\insertnavigation{\paperwidth}\vskip5pt
  \end{beamercolorbox}%
  \begin{beamercolorbox}[colsep=1.5pt]{lower separation line head}
  \end{beamercolorbox}
}

% --- 自定义列表 ---
\setbeamertemplate{itemize items}[ball]
\setbeamertemplate{enumerate items}[ball]

\begin{document}

\begin{frame}
  \titlepage
\end{frame}

\begin{frame}{目录}
  \small
  \linespread{0.8}\selectfont
  \tableofcontents[hideallsubsections]
\end{frame}

\section[基金选取]{基金选取说明}
% --- Fund Selection Slide ---
\begin{frame}{研究基金选择与说明}
  \footnotesize
  本小组研究基金选择说明如下:
  \begin{table}
    \centering
    \renewcommand{\arraystretch}{1.2} 
    \setlength{\tabcolsep}{6pt}       
    \resizebox{\textwidth}{!}{
      \begin{tabular}{p{0.12\textwidth} p{0.40\textwidth} p{0.40\textwidth}}
        \toprule[1.5pt]
        \textbf{维度} & \textbf{主分析对象} & \textbf{对照组对象} \\
        \midrule
        
        \textbf{基金名称} & \textbf{中欧时代先锋股票A (001938)} & \textbf{中欧趋势LOF (166001)} \\
        \rowcolor{gray!10} 
        \textbf{基金经理} & 周应波 (主要研究其任职期) & 周应波 (主要研究其任职期) \\
        
        \textbf{选取理由} & 
        \begin{itemize}
            \item 历史业绩卓越,2021年获“金牛奖”;
            \item 主动管理型成长风格基金,成立以来最大盈利达+193.84\%,平均年化回报27\%;
            \item 5年以上历史数据完整。
        \end{itemize} 
        & 
        \begin{itemize}
            \item 由同一位基金经理管理;
            \item 用于验证经理投资风格的一致性。进行对比分析;
            \item 对比不同产品定位下的业绩差异。
        \end{itemize} \\
        
        \rowcolor{gray!10} 
        \textbf{研究区间} & 2019.01.01 - 2025.11.21 & 2019.01.01 - 2025.11.21 \\
        \bottomrule[1.5pt]
      \end{tabular}
    }
  \end{table}
\end{frame}


\section[因子计算]{因子计算结果}


\section[主基金分析]{主基金业绩分析:多因子模型}
% ==============================================================
% Part 3: Main Fund Performance Attribution (Fixed Top Table Layout)
% ==============================================================
\newcommand{\RegressionTable}{
  \begin{table}
    \centering
    \renewcommand{\arraystretch}{1.0}
    \setlength{\tabcolsep}{3pt} 
    \scriptsize
    \vspace{-1em}
    \caption{中欧时代先锋股票A(001938) 多模型回归结果汇总}
    \vspace{-1.5em}
    \begin{tabular}{l c c c c c c c c}
      \toprule[1pt]
      \textbf{模型} & \textbf{MKT} & \textbf{SMB} & \textbf{HML} & \textbf{UMD} & \textbf{R\_pead} & \textbf{R\_fin} & \textbf{MKT\textsuperscript{2}} & \textbf{Alpha(年化)} \\
      \midrule
      \textbf{CAPM} & 0.65 & - & - & - & - & - & - & 2.81\% \\
      \rowcolor{gray!10}
      \textbf{Carhart} & 0.57 & -0.06 & \textbf{-0.26} & \textbf{0.18} & - & - & - & 3.66\% \\
      \textbf{DHS} & 0.56 & -0.11 & -0.20 & 0.18 & \textbf{0.08} & -0.13 & - & 4.75\% \\
      \rowcolor{gray!10}
      \textbf{TM择时} & 0.56 & -0.07 & -0.26 & 0.18 & - & - & \textbf{-0.88} & \textbf{6.62\%} \\
      \bottomrule[1pt]
    \end{tabular}
  \end{table}
  \vspace{-0.5em} % 调整表格与下方内容的间距
}

% --- Slide 1: Risk & Style ---
\begin{frame}{主基金归因:基础多因子模型}
  \footnotesize
  \RegressionTable
  \begin{block}{\textbf{基础模型解读 (CAPM \& Carhart)}}
    \begin{itemize}
        \item \textbf{低系统性风险:} 市场 $\beta$ (MKT) 始终维持在 \textbf{0.57-0.65} 区间,显著低于 1,说明基金在市场下跌时具有较好的防御性。
        \item \textbf{投资风格分析:} 
        \begin{itemize}
          \footnotesize
            \item \textbf{成长型:} HML 系数为 \textbf{-0.26} (显著负),偏好高估值成长股;
            \item \textbf{大盘股:} SMB 系数为负,偏好大盘蓝筹;
            \item \textbf{动量效应:} UMD 系数为 \textbf{0.18} (显著正),善于捕捉市场上涨趋势。
        \end{itemize}
    \end{itemize}
  \end{block}
\end{frame}

% --- Slide 2: Behavioral Factors ---
\begin{frame}{主基金归因:行为金融因子纳入}
  \RegressionTable
  \footnotesize
  \begin{block}{\textbf{行为金融模型解读 (DHS)}}
    \begin{itemize}
        \item \textbf{利用市场错误定价:} 
        \begin{itemize}
          \footnotesize
            \item 引入 \textit{盈余惯性(R\_pead)} 后,系数显著为正 (\textbf{0.08});
            \item 说明经理善于挖掘业绩超预期、但市场尚未充分反应的股票。
        \end{itemize}
        \item \textbf{Alpha 提升:} 
        \begin{itemize}
          \footnotesize
            \item 模型解释力增强后,年化 Alpha 从 3.66\% 提升至 \textbf{4.75\%};
            \item 表明部分超额收益来源于对投资者行为偏差(如反应不足)的利用。
        \end{itemize}
    \end{itemize}
  \end{block}
\end{frame}

% --- Slide 3: Timing vs Selection ---
\begin{frame}{主基金归因:择时与选股能力剥离}
  \RegressionTable
  \footnotesize
  \begin{block}{\textbf{择时能力模型解读 (TM)}}
    \scriptsize
    \begin{itemize}
      \item \textbf{“负”择时现象:} 择时系数 $\gamma$ (MKT\textsuperscript{2}) 为 \textbf{-0.88},说明从模型上看,TM模型认为基金经理的择时能力较弱。但经过深入分析,我们认为这主要反映了基金经理偏好左侧交易(越跌越买)。
      \item \textbf{真实的选股能力:} 剥离左侧交易的择时负面损耗后,纯选股 Alpha 飙升至 \textbf{6.62\%};说明该基金的核心竞争力在于\textcolor{red}{卓越的选股能力},而非仓位择时。
      \item \textbf{公墓基金底仓限制:} 考虑到本产品为时代先锋股票型基金,合同要求股票投资占基金资产的比例为80\%–95\%,因此基金经理无法在熊市通过大幅减仓来做“正择时”。
    \end{itemize}
  \end{block}
\end{frame}


\section[对照基金分析]{对照基金业绩分析:多因子模型与基金经理风格画像}
% ==============================================================
% Part 4: 对照基金业绩归因分析与经理风格画像
% ==============================================================
% --- Slide 4.1: Control Group Attribution ---
\begin{frame}{对照基金归因:稳健型产品的风格验证}
  \begin{table}
    \centering
    \vspace{-1em}
    \caption{中欧趋势 LOF (166001) 多模型回归结果汇总}
    \vspace{-1.3em}
    \resizebox{1\textwidth}{!}{%
      \renewcommand{\arraystretch}{1.0}
      \setlength{\tabcolsep}{4pt}
      \begin{tabular}{l c c c c c c c c}
        \toprule[1pt]
        \textbf{模型} & \textbf{MKT} & \textbf{SMB} & \textbf{HML} & \textbf{UMD} & \textbf{R\_pead} & \textbf{R\_fin} & \textbf{MKT\textsuperscript{2}} & \textbf{Alpha} \\
        \midrule
        \textbf{CAPM} & \textbf{0.49} & - & - & - & - & - & - & 1.67\% \\
        \rowcolor{gray!10}
        \textbf{Carhart} & 0.46 & -0.08 & \textbf{-0.14} & 0.08 & - & - & - & 2.30\% \\
        \textbf{DHS} & 0.46 & -0.08 & -0.15 & 0.08 & \textbf{0.05} & 0.03 & - & 1.94\% \\
        \rowcolor{gray!10}
        \textbf{TM择时} & 0.46 & -0.09 & -0.14 & 0.08 & - & - & \textbf{-0.57} & \textbf{4.23\%} \\
        \bottomrule[1pt]
      \end{tabular}%
    }
  \end{table}

  \vspace{-0.3cm}
  
  % Block 标题加粗,内容字号缩小
  \begin{block}{\textbf{关键数据解读}}
    \footnotesize % 字号调整为 footnotesize
    \begin{itemize}
        \item \textbf{低风险定位:} 市场 Beta (0.49) 低于主基金 (0.65),验证了其防御属性。
        \item \textbf{风格延续:} 同样呈现 \textbf{负HML} (成长) 和 \textbf{负SMB} (大盘) 特征,但幅度较主基金温和。
        \item \textbf{择时与行为因子:} 
        \begin{itemize}
            \item \footnotesize 
            盈余惯性因子(R\_pead) 暴露为正,说明在稳健型产品中经理依然使用了同样的选股逻辑。
            \item \footnotesize 择时系数为 \textbf{-0.57} (显著负),剥离后 Alpha 显著回升至 4.23\%。
        \end{itemize}
    \end{itemize}
  \end{block}
\end{frame}

% --- Slide 4.2: Comparison ---
\begin{frame}{两支基金对照:进攻 vs 防守}
  \small
  \textbf{主基金 (001938) 与 对照基金 (166001) 核心指标对比}
  \begin{table}
    \centering
    \caption{双基金风险收益特征对比表}
    \vspace{-1em}
    \resizebox{1.0\textwidth}{!}{%
      \renewcommand{\arraystretch}{1.4}
      \begin{tabular}{l | c | c | l}
        \toprule[1.5pt]
        \textbf{维度} & \textbf{\textcolor{red}{时代先锋 (主)}} & \textbf{\textcolor{blue}{中欧趋势 (对照)}} & \textbf{差异解读} \\
        \midrule
        \rowcolor{gray!10} 
        \textbf{定位} & \textbf{进攻型旗舰} & \textbf{稳健防御型} & 产品风险收益特征定位不同 \\
        
        \textbf{市场 Beta} & \textbf{0.65} & \textbf{0.49} & 对照组仓位控制更严,波动更小 \\
        \rowcolor{gray!10} 
        \textbf{成长暴露(HML)} & \textbf{-0.26} & -0.14 & 主基金的成长风格更为极致 \\
        
        \textbf{动量暴露(UMD)} & \textbf{0.18} & 0.08 & 主基金更积极捕捉市场热点趋势 \\
        \rowcolor{gray!10} 
        \textbf{年化 Alpha} & \textbf{3.66\%} & 2.30\% & 高风险敞口换取了更高的超额收益 \\
        \bottomrule[1.5pt]
      \end{tabular}%
    }
  \end{table}
  \footnotesize \textit{*注:Alpha与因子数据基于 Carhart 四因子模型结果。}
\end{frame}

% --- Slide 4.3: Manager Profile ---
\begin{frame}{基金经理投资风格画像}
  \small
  基于两只基金的量化归因,我们勾勒出基金经理的投资画像:
  \begin{columns}[T]
    \column{0.48\textwidth}
      % Block 标题加粗,内容字号缩小
      \begin{block}{\textbf{1. 核心能力圈 (不变)}}
        \footnotesize 
        \begin{itemize}
            \item \textbf{大盘成长捕手:} 始终坚持“大盘+成长”的选股底色 (负SMB, 负HML)。
            \item \textbf{行为金融套利:} 善于利用“盈余惯性”挖掘被低估的成长股。
            \item \textbf{选股驱动:} 超额收益主要源自选股,而非仓位择时。
        \end{itemize}
      \end{block}

    \column{0.48\textwidth}
      % Block 标题加粗,内容字号缩小
      \begin{block}{\textbf{2. 差异化驾驭能力 (变)}}
        \footnotesize
        \begin{itemize}
            \item \textbf{风险敞口管理:} 
            \item \textcolor{red}{001938 (矛)}:放宽 Beta 至 0.65,极致暴露动量因子,追求进攻;
            \item \textcolor{blue}{166001 (盾)}:压降 Beta 至 0.49,平滑波动,追求绝对收益。
        \end{itemize}
      \end{block}
  \end{columns}
  
  \vspace{0.5cm}
  \begin{itemize}
    \item \textbf{总结:在统一的成长股投资框架下,通过精准的 Beta 管理实现了产品的差异化定位。}
  \end{itemize}
\end{frame}



\section[持仓模拟]{主基金持仓模拟与动态归因分析}

\begin{frame}
  \begin{itemize}
    \item 等待安愉老师计量归来填充内容。
  \end{itemize}
\end{frame}


\section[总结]{总结与小组分工}

\begin{frame}{小组分工}
  \small
  \begin{columns}
    \begin{column}{1\textwidth}
      \begin{block}{小组成员分工}
        \begin{itemize}
          \item 董安愉: 
          \item 李想: 
          \item 杨晨艺: 完成主基金多因子分析和对照因子的多因子分析(对应作业文件2-4题和第6题)
        \end{itemize}
      \end{block}
    \end{column}
  \end{columns}

  \vspace{1em}

  \centering
    \footnotesize \textit{-- Thank you for your Attention--} \footnote{This presentation was made using \LaTeX{} and Beamer} \\
    \footnotesize \textit{-- End of Report --}
\end{frame}

\end{document}